\documentclass[10pt,a4paper]{article}
\author{Yaroslav Mykoliv}
\title{EVENT CAMERA PAPER SUMMARIES}
\begin{document}
\maketitle
%\tableofcontents

\section{Event-based Vision: A Survey}
\paragraph{}
Event cameras don't capture images at a fixed rate, instead they measure per-pixel brightness changes asynchronously.
The results is a stream of events with information about time, location and sign of the brightness changes.

The advantages of the event cameras over the traditional ones are: 
\begin{itemize}
\item high temporal resolution, no motion blur
\item high dynamic range
\item low power consumption
\item no latency
\end{itemize}
\subsection{}

\section{Events-to-Video: Bringing Modern Computer Vision to Event Cameras}

\section{A QVGA 143 dB Dynamic Range Frame-Free PWM Image Sensor With Lossless Pixel-Level Video Compression and Time-Domain CDS}
\paragraph{}
Presents ATIS (asynchronous time-based image sensor), which combines event- and frame-based (exposure) information.
In short, each pixel is a combination of DVS-like change detector and a conditional exposure measurement unit.

It achieves highly efficient sensor-driven ideally lossless video compression, by suppressing temporal redundancy using asynchronous pixel change detections.
Subsequently, an asynchronous event-based communication scheme (Address Even Representation) is used in order to provide efficient allocation of the transmition channel.

Ideally, the redundant data is not recorded at all. In practice, however, due to static background noise is usually present, which limits achievable video compression factor.
Without considering noise, compression factors depend only on scene dynamics.

\section{Real-Time, High-Speed Video Decompression Using a Frame- and Event-based DAVIS Sensor}
Considers event data as natively-compressed.
Proposes first real-time event decompression algorithm for the DAVIS.
Events are processed one by one, not in batches.
Essentially this paper considers reconstruction problem in decompression scenario.
DAVIS produces an asynchronous stream of events which encode changes in the pixel brightness and at any point in time an exposure can be started, resulting in the output of a standard still image.

The proposed method for decompression relies on integration of subsequent intensity steps on arrival of each event (in log-domain).
This leads to an error build-up from brightness update per event originating from the inter-pixel mismatch of the ON/OFF thresholds.
The counter measure against this error build-up the sampled still images are used in the following way:
1. The event update steps are updated according to the decompression error of the resulting decompressed image and the next still image
2. The reconstructions starts anew from the still image, preventing the error buildup (decompression error is reset)

Main drawback: the high reconstruction error even in-between subsequent still images.

\section{High-DR Frame-Free PWM Imaging with asynchronous AER Intensity Encoding and Focal Plane Temporal Redundancy Suppression}
Presents asynchronous time-based image sensor (ATIS).
Contains nice outline of motivation of the sensor and its concept.

Argues that the sensor natively realizes optimal lossless pixel-level video compression through temporal redundancy suppression.

\section{Real-Time, Near-Lossless, Energy-Constrained Compression Method for High Frame Rate Videos}
Not a very good paper that mentions DAVIS, but has little to do with it.

Presents a simple video compression method, containing predictor (next frame is predicted to be the same as the previous one),
quantizer (3 different quantizers are considered) and entropy coder (Huffman).

The interesting part is that they pick the parameters for the model by energy minimization, which contains 3 terms: distorion, bitrate and computation effort.
Instead of gradient descent, a weird parameter search algorithm is used. No notion of well posedness or optimality of the soulition are considered.

TODO: read the referenced paper, they took the optimization method from

\section{Asynchronous Spatial Image Convolutions for Event Cameras}
\paragraph{}
Proposes a method to compute the convolution of a linear spatial kernel with the ouput of an event camera which directly operates on the evenet stream without the need for reconstruction of intensity frames.
An internal state that encodes the convolved image information is used. Each of its pixels carries a timestamp of the last event that updated that pixel along with the latest state information.
The internal state can be separately read off as often as and whenever required.

For each pixel the exact analytic solution to the associated ordinary differential equation of the filter in continuous time is computed and evaluated at descrete time instances.
The process requires integration over events, which produces an accumulated error from quantisation and sensor noise.
This in turn results in a drift and undermines low temporal-frequency components of the convolution estimate over time.
However, it is argued that only high temporal-frequency information is relevant.

To this end, a high-pass filter in the frequency domain is imployed, which is implemented in the time domain via inverse Laplace transform, which requires solving a constant coefficient linear differential equation explicitly.
An asynchronous distributed pixel-by-pixel update is derived to compute the filter state.

It is claimed that a straightforward generalisation of the filter equations allows to compute image pyramids as well.

Authours mention possibilities to alternative feature states, continuous-time optical flow state, and application of event=based convolutions to convolutional neural networks.

TODO: read up on Laplace transform; how is solution to the ODE in the paper derived?

\section{Time-Based Compression and Classification of Heartbeats}
Proposes classification method for ECG recordings on samples from the data (time-based representation) and not from reconstructions.

The output from the IF sampler, used in the paper, probably can be interpreted as a signal from a single pixel of an event camera.
The samples provide a compressed representation for signals in which information is localized in high-amplitude transients overlaid on low-amplitude background noise.

A number of reconstruction algorithms for such data is mentioned.
The authors have also shown in a previous work that the reconstruction error can be bounded in bandlimited spaves and perfect recovery is possible in finite-dimensional spaces, as long as the sample contraints are satisfied.
Additionaly, a number of different representations of the data are mentioned.

\section{Continuous-time Intensity Estimation Using Event Cameras}
!Code available!

A continuous formulation of event-based intensity estimation using complementary filtering to combine image frames with events is presented, using the concept of a continuous-time image state that is asynchronously updated with every event.
Rather than resetting the instensity estimate with arrival of a new frame, the formulation retains the high-dynamic-range information from events.

Events are continuous-time signals even though they are not continuous functions of time; the time variable on which they depend varies continuously.

A complementary filter is used to to fuse the event field with log-intensity frames. 
Such filters are ideal for fusing signals that have complementar frequency noise characteristics.
Due to high-temporal resolution of events, they provide reliable high-frequency information, but their integration amplifies low-frequency disturbance (drift).
Clasical image frames have poor high-frequency fidelity, but provide reliable low-frequency reference intensity information.
The proposed asynchronous complementary filter architecture combines a high-pass version of event stream with low-pass version of classical frames.

The proposed filter is a continuous-time ordinary differential equation. The result is derived step-by-step as solutions to this equation at each of the timestamps. 
The solution for previous time step is used as initial condition for the next time-interval.

Inside the ODE there is a parameter that controls the relative information contributed by image frames or events. 
Authors propose to replace it with a function to dynamically adjust the relative dependence on image frames or events, which can be useful when image frames are compromised.
The heuristic that pixels reporting an intensity close to the minimum or maximum output of the camera may be compromised is used.

The proposed method works on both DAVIS-like data and on pure event stream and can be used to augment other methods.

\section{Event-driven Video Frame Synthesis}
!Code is available!





\end{document}