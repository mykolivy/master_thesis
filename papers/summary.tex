\documentclass[10pt,a4paper]{article} \author{Yaroslav Mykoliv} \title{EVENT
CAMERA PAPER SUMMARIES} \begin{document} \maketitle
%\tableofcontents

\section{Event-based Vision: A Survey} \paragraph{} Event cameras don't capture
images at a fixed rate, instead they measure per-pixel brightness changes
asynchronously.  The results is a stream of events with information about time,
location and sign of the brightness changes.

The advantages of the event cameras over the traditional ones are:
\begin{itemize} \item high temporal resolution, no motion blur \item high
dynamic range \item low power consumption \item no latency \end{itemize}
	\subsection{}

The paper is a survey on the current problems and applications of event-based
vision, which probably requires novel computer vision algorithms. 

\section{Events-to-Video: Bringing Modern Computer Vision to Event Cameras}
!Implemented in PyTorch, but so far I failed to find the source code!

In this paper authors establish a bridge between vision with event cameras and
conventional computer vision.  A U-Net network architecture is used to
reconstruct natural videos from a stream of events, trained on a large quantity
of simulated event sequences with corresponding ground-truth images.  After
this, off-the-shelf computer vision algorithms that were built to process
conventional images are applied for object classification and visual-inertial
odometry problems.

It is noted that events cannot be directly integrated to recover accurate
intensity images in practice.  Events are represented as a spatio-temporal voxel
grid, which introduces some latency due to processing of events in windows.

The training data was generated from MS-COCO images by first mapping those to a
3D plane, with events triggered by random camera motion within the scene. The
network is trained on simulated data, but is shown to generalize well to real
event data.

The network itself features residual blocks as well as a recurrent connection to
propagate intensity  information forward in time.  It does not need to
reconstruct a new image from scratch at every time step, but only to
incrementally update the previous reconstructions using the new sequence of
events.

The objective is to minimize the calibrated perceptual loss (LPIPS), which
averages the distances between VGG features computed across multiple layers. By
minimizing this loss, the network essentially learns to endow the reconstructed
images with natural statistics. It is important to note that instead of focusing
mainly on the quality of the reconstructions, the authors build their approach
with the goal of applying standard computer vision techniques to the
reconstructions.

\section{A QVGA 143 dB Dynamic Range Frame-Free PWM Image Sensor With Lossless
Pixel-Level Video Compression and Time-Domain CDS} \paragraph{} Presents ATIS
(asynchronous time-based image sensor), which combines event- and frame-based
(exposure) information.  In short, each pixel is a combination of DVS-like
change detector and a conditional exposure measurement unit.

It achieves highly efficient sensor-driven ideally lossless video compression,
by suppressing temporal redundancy using asynchronous pixel change detections.
Subsequently, an asynchronous event-based communication scheme (Address Even
Representation) is used in order to provide efficient allocation of the
transmission channel.

Ideally, the redundant data is not recorded at all. In practice, however, due to
static background noise is usually present, which limits achievable video
compression factor.  Without considering noise, compression factors depend only
on scene dynamics.

\section{Real-Time, High-Speed Video Decompression Using a Frame- and
Event-based DAVIS Sensor} Considers event data as natively-compressed.  Proposes
first real-time event decompression algorithm for the DAVIS.  Events are
processed one by one, not in batches.  Essentially this paper considers
reconstruction problem in decompression scenario.  DAVIS produces an
asynchronous stream of events which encode changes in the pixel brightness and
at any point in time an exposure can be started, resulting in the output of a
standard still image.

The proposed method for decompression relies on integration of subsequent
intensity steps on arrival of each event (in log-domain).  This leads to an
error build-up from brightness update per event originating from the inter-pixel
mismatch of the ON/OFF thresholds.  The counter measure against this error
build-up the sampled still images are used in the following way: 1. The event
update steps are updated according to the decompression error of the resulting
decompressed image and the next still image 2. The reconstructions starts anew
from the still image, preventing the error buildup (decompression error is
reset)

Main drawback: the high reconstruction error even in-between subsequent still
images.

\section{High-DR Frame-Free PWM Imaging with asynchronous AER Intensity Encoding
and Focal Plane Temporal Redundancy Suppression} Presents asynchronous
time-based image sensor (ATIS).  Contains nice outline of motivation of the
sensor and its concept.

Argues that the sensor natively realizes optimal lossless pixel-level video
compression through temporal redundancy suppression.

\section{Real-Time, Near-Lossless, Energy-Constrained Compression Method for
High Frame Rate Videos} Not a very good paper that mentions DAVIS, but has
little to do with it.

Presents a simple video compression method, containing predictor (next frame is
predicted to be the same as the previous one), quantizer (3 different quantizers
are considered) and entropy coder (Huffman).

The interesting part is that they pick the parameters for the model by energy
minimization, which contains 3 terms: distortion, bitrate and computation
effort.  Instead of gradient descent, a weird parameter search algorithm is
used. No notion of well posedness or optimality of the solution are considered.

TODO: read the referenced paper, they took the optimization method from

\section{Asynchronous Spatial Image Convolutions for Event Cameras} \paragraph{}
Proposes a method to compute the convolution of a linear spatial kernel with the
output of an event camera which directly operates on the event stream without
the need for reconstruction of intensity frames.  An internal state that encodes
the convolved image information is used. Each of its pixels carries a timestamp
of the last event that updated that pixel along with the latest state
information.  The internal state can be separately read off as often as and
whenever required.

For each pixel the exact analytic solution to the associated ordinary
differential equation of the filter in continuous time is computed and evaluated
at discrete time instances.  The process requires integration over events, which
produces an accumulated error from quantisation and sensor noise.  This in turn
results in a drift and undermines low temporal-frequency components of the
convolution estimate over time.  However, it is argued that only high
temporal-frequency information is relevant.

To this end, a high-pass filter in the frequency domain is employed, which is
implemented in the time domain via inverse Laplace transform, which requires
solving a constant coefficient linear differential equation explicitly.  An
asynchronous distributed pixel-by-pixel update is derived to compute the filter
state.

It is claimed that a straightforward generalisation of the filter equations
allows to compute image pyramids as well.

Authours mention possibilities to alternative feature states, continuous-time
optical flow state, and application of event=based convolutions to convolutional
neural networks.

TODO: read up on Laplace transform; how is solution to the ODE in the paper
derived?

\section{Time-Based Compression and Classification of Heartbeats} Proposes
classification method for ECG recordings on samples from the data (time-based
representation) and not from reconstructions.

The output from the IF sampler, used in the paper, probably can be interpreted
as a signal from a single pixel of an event camera.  The samples provide a
compressed representation for signals in which information is localized in
high-amplitude transients overlaid on low-amplitude background noise.

A number of reconstruction algorithms for such data is mentioned.  The authors
have also shown in a previous work that the reconstruction error can be bounded
in band-limited spaces and perfect recovery is possible in finite-dimensional
spaces, as long as the sample constraints are satisfied.  Additionaly, a number
of different representations of the data are mentioned.

\section{Continuous-time Intensity Estimation Using Event Cameras} !Code
available!

A continuous formulation of event-based intensity estimation using complementary
filtering to combine image frames with events is presented, using the concept of
a continuous-time image state that is asynchronously updated with every event.
Rather than resetting the intensity estimate with arrival of a new frame, the
formulation retains the high-dynamic-range information from events.

Events are continuous-time signals even though they are not continuous functions
of time; the time variable on which they depend varies continuously.

A complementary filter is used to to fuse the event field with log-intensity
frames.  Such filters are ideal for fusing signals that have complementary
frequency noise characteristics.  Due to high-temporal resolution of events,
they provide reliable high-frequency information, but their integration
amplifies low-frequency disturbance (drift).  Classical image frames have poor
high-frequency fidelity, but provide reliable low-frequency reference intensity
information.  The proposed asynchronous complementary filter architecture
combines a high-pass version of event stream with low-pass version of classical
frames.

The proposed filter is a continuous-time ordinary differential equation. The
result is derived step-by-step as solutions to this equation at each of the
timestamps.  The solution for previous time step is used as initial condition
for the next time-interval.

Inside the ODE there is a parameter that controls the relative information
contributed by image frames or events.  Authors propose to replace it with a
function to dynamically adjust the relative dependence on image frames or
events, which can be useful when image frames are compromised.  The heuristic
that pixels reporting an intensity close to the minimum or maximum output of the
camera may be compromised is used.

The proposed method works on both DAVIS-like data and on pure event stream and
can be used to augment other methods.

\section{Event-driven Video Frame Synthesis} !Code available!

A new high framerate video synthesis framework by fusing intensity frames with
event streams is introduced.  It contains two steps: Differentiable Model-based
Reconstruction (DMR) and Residual Denoiser (DR).

Reconstruction is performed by DMR, which fuses event- and frame-based
information.  It's results may have visual artifacts due to the ill-posedness of
the fusion problem and different noise levels between the two sensing
modalities.  To address this, additional DR deep-nn model is added on top to
denoise the result.

The DMR is performed by minimizing a weighted combination of several loss
functions: pixel loss, which includes per-pixel difference against intensity and
event pixels in l1 norm over the entire available data range; and sparsity loss
- total variation (TV) sparsity in the spatial and temporal dimensions of the
resulting high-res tensor.  Sparsity loss has two terms: one denoising term for
intensity tensor and another for event denoising.

For RD authors enforce generated event frames to contain less than 20\% of
events, which is motivated by a statistical analysis from supplementary material
of the paper.

The event sensing model requires binning events into frames.  Events happening
at different locations but at very close timestamps can be processed in the same
event frame.  Two binning strategies are explored.

The authors mention that existing DAVIS dataset doesn't contain enough sharp
intensity images captured at high speed for training/fine tuning.  Because of
that they are planning to use event simulation.

!The paper is a  treasure trove for paper references!

\section{A Low-complexity Image Compression Algorithm for Address-Event
Representation (AER) PWM Image Sensors}

In this paper a block based image compression algorithm, which exploits the
intensity-ordered nature of AER image sensors to simultaneously achieve lower
computational complexity and lower address overhead for individual pixels.
Under this scheme, the address vector overhead can be dramatically reduced.

The pixels trigger in the descending order of their sensed light intensity.

The scheme is as follows: the image is first partitioned into non=overlapping
4x4 pixel blocks (Coded Pixel Blocks, CPB).  For each block a pseudo-gradient is
computed and if it is higher than a threshold, the block is EP (Edge Patterns),
otherwise -- UP (Uniform Patterns).  In case of EP block 1 bit for each pixel is
sent along with mean and G.  For UP blocks only the mean is sent. A flag bit is
used to indicate whether a block is EP or UP.

The raw image is assembled as either a uniform block with the mean value or the
sum of a uniform block and the bipolar bit-pattern, B, scaled by the
pseudo-gradient coefficient, G. Final decoded image is obtained by performing a
simple 3-by-3 averaging filter over the raw image.

Redundancy in the address vector is suppressed with several CPBs being sent
together with their relative addresses coded by run-length coding.

The scheme's performance improves for images with less high frequency content at
its global level.

\section{The Event-Camera Dataset and Simulator: Event-based Data for Pose
Estimation, Visual Odometry and SLAM} A collection of datasets captured with a
DAVIS in a variety of synthetic and real environments.  In addition to
global-shutter intensity images and asynchronous events, inertial measurements
and ground-truth camera poses from a motion-capture system are provided.

Furthermore, an open-source simulator to create synthetic event-camera data is
described.

\section{High Speed and High Dynamic Range Video with Event Camera} !Code
available!

Basically this is a good re-write of the "Events-to-Video: Bringing Modern
Computer Graphics to Event Cameras" paper.

Contains more information on network architecture, post processing, color video
reconstruction and applications.


\end{document}
